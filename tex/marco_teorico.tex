\documentclass[../main/main.tex]{subfiles}

\begin{document}
\espacio

  %------------ TODO ------------%
  En este capítulo se muestra el marco teórico TODO.

  \section{Antecedentes}

  \subsection{Computación paralela}

  La computación paralela es una rama de la informática que se encarga del estudio de la ejecución de una tarea dividida en sub-procesos o varias tareas independientes de forma simultánea en forma de hilos de ejecución en un grupo de procesadores llamados también procesadores multinúcleo, que luego de realizar dichas tareas sincronizan sus resultados a fin de mantener la integridad de los datos.

  Los microprocesadores actuales contienen comúnmente dos tipos de núcleos, los núcleos físicos y los núcleos lógicos. El zócalo de una tarjeta madre contiene un microprocesador, este contiene uno o más núcleos físicos; un núcleo físico es aquel que se encuentra físicamente dentro del circuito integrado del microprocesador, mientras que un núcleos lógico es una porción de un núcleo físico. Las tareas son asignadas a los núcleos físicos; estas tareas pueden dividirse en tareas más pequeñas a fin de resolver un gran problema en partes pequeñas que al final serán unidas para generar la solución, estas partes pequeñas son llamadas ``hilos'' y son las que se ejecutan en los núcleo lógicos.



  \bibliografia
\end{document}