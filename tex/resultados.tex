\documentclass[../main/main.tex]{subfiles}

\begin{document}
\espacio
  En este capítulo se muestran la metodología de investigación empleada en el trabajo.

  \section{Metodología de investigación}

  El método de enfoque para el desarrollo del proyecto es el método cuantitativo, ya que los resultados se demuestran con tablas comparativas recopiladas de las pruebas a ser ejecutadas en ambos ambientes de procesamiento. El instrumento de medición será un temporizador por software iniciado al momento de la ejecución de cada prueba y finalizado al terminar el procesamiento del algoritmo.
  
  A pesar de que el proceso tiene bases bibliográficas de acuerdo a los antecedentes; en este caso, los resultados aceptados serán los que provengan del proceso formal empírico, ya que se ejecutarán programas de cómputo para llegar a los resultados.
  
  El grado de abstracción es el de una investigación aplicada, ya que su principal objetivo se basa en resolver el problemas práctico con un margen de generalización limitado de acuerdo a los límites propuestos.

  El método científico aplicado es el racionalista, ya que los resultados se toman en cuenta sin ninguna interpretación previa.
  
  El enfoque es positivista, ya que se desea demostrar la aserción del incremento de velocidad de ejecución del algoritmo AES Rijndael en GPU con respecto a la velocidad de ejecución en CPU.

\end{document}