% !TeX root = ../main/main.tex
\documentclass[../main/main.tex]{subfiles}

\begin{document}

  \begin{center}
    \textbf{\large Resumen}
  \end{center}

  El presente estudio tiene el proposito de demostrar el incremento en la velocidad de ejecución del algoritmo AES Rijndael ejecutado en GPU con respecto a la ejecución del mismo en la CPU, mediante el paralelismo de hilos y tareas.

  Para demostrar la validez de dicho incremento en la velocidad de ejecución del algoritmo se realizaron pruebas mediante ``Scripts'' ejecutados en el entorno de desarrollo Python, los cuales generaron los resultados mostrados en los capítulos subsecuentes.

  Posterior a las pruebas realizadas, se llega a la conclusión de que la velocidad en ejecución del algoritmo AES Rijndael se incrementa de manera significativa cuando el algoritmo es ejecutado en la GPU, pero por limitaciones de la tecnología actual de bus de transferencia de datos, la diferencia de tiempo de ejecución en GPU es opacada por el tiempo necesario para la transferencia de datos desde y hacia la GPU.

  En base al presente estudio se puede concluir que actualmente existen dispositivos capaces de realizar cálculos densos en cuestión de fracciones de segundo, pero por la restricción física de las líneas de transmisión de datos actuales, los resultados se llegan a obtener en un tiempo que supera hasta en diez veces al tiempo utilizado para el cálculo en la CPU.

  No obstante, al igual que se logró el cambio de paradigma de procesadores de 32 bits a 64 bits, actualmente se están investigando medios de transferencia que no se limitan por las pistas de alguna aleación o de metal, esta tecnología es conocida como computación cuántica y sin duda alguna se encontrará disponible para el público en general en un futuro cercano.

\end{document}