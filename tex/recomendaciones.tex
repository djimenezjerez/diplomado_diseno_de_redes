% !TeX root = ../main/main.tex
\documentclass[../main/main.tex]{subfiles}

\begin{document}
\espacio

  \begin{itemize}[noitemsep,nolistsep]
    \item La alternativa utilizada para la paralelización del algoritmo en esta investigación fue la librería Numba en el entorno Python que es un entorno de alto nivel; se recomienda realizar la paralelización del algoritmo en entornos de bajo nivel como C o C++ y comparar los resultados obtenidos en esta investigación para concluir con mayor certeza la aceleración del algoritmo ejecutado en GPU.
    \item En esta investigación se trabajó con el algoritmo AES Rijndael; se recomienda la ejecución de otros algoritmos de encriptación, simétricos y asimétricos, como IDEA o DES para comparar los resultados de la aceleración en ejecución de algoritmos simétricos con respecto a la ejecución de algoritmos asimétricos.
    \item Dado el tiempo inherente de transferencia de datos entre la CPU y la GPU, se recomienda realizar la investigación de asignación de tareas a la GPU enfocada en la obtención de resultados con un método mas veloz en cuanto a la velocidad de bus, o bien no hacer uso de ningún bus y realizar las operaciones directamente en la GPU sin pasar por la CPU.
  \end{itemize}


\end{document}